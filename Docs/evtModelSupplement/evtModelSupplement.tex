%% Default Latex document template
%%
%%  blake@rcs.ee.washington.edu

\documentclass[letterpaper]{article}

% Uncomment for bibliog.
%\bibliographystyle{unsrt}

\usepackage{graphicx}
\usepackage{lineno}
\usepackage{hyperref}
%\usepackage{fancyhdr}

%%%%%%%%%%%%%%%%%%%%%%%%%%%%%%%%%%%%%%%%5
%
%  Set Up Margins
\input{templates/pagedim.tex}

%
%        Font selection
%
%\renewcommand{\rmdefault}{ptm}             % Times
%\renewcommand{\rmdefault}{phv}             % Helvetica
%\renewcommand{\rmdefault}{pcr}             % Courier
%\renewcommand{\rmdefault}{pbk}             % Bookman
%\renewcommand{\rmdefault}{pag}             % Avant Garde
%\renewcommand{\rmdefault}{ppl}             % Palatino
%\renewcommand{\rmdefault}{pch}             % Charter


%%%%%%%%%%%%%%%%%%%%%%%%%%%%%%%%%%%%%%%%%%%%%%%%%
%
%         Page format Mods HERE
%
%Mod's to page size for this document
\addtolength\textwidth{0cm}
\addtolength\oddsidemargin{0cm}
\addtolength\headsep{0cm}
\addtolength\textheight{0cm}
%\linespread{0.894}   % 0.894 = 6 lines per inch, 1 = "single",  1.6 = "double"

% header options for fancyhdr

%\pagestyle{fancy}
%\lhead{LEFT HEADER}
%\chead{CENTER HEADER}
%\rhead{RIGHT HEADER}
%\lfoot{Hannaford, U. of Washington}
%\rfoot{\today}
%\cfoot{\thepage}



% Make table rows deeper
%\renewcommand\arraystretch{2.0}% Vertical Row size, 1.0 is for standard spacing)

\begin{document}
\setpagewiselinenumbers        %  Line numbers for edits to drafts.
\modulolinenumbers[1]          %  number every N lines

% \linenumbers                   %  start numbering lines here


\date{\today}

\section*{4x4 State Transition Criteria}
First we define Drive pressure, $drivepress,$
as the pressure in the compartment which drives eversion.
This is compartment 1 for the 1-compartment model and compartment 2 for the 2-compartment
model.  We then test some state variables to determine four boolean
state transition  conditions:
\begin{verbatim}
        LoPress = drivepress < Pth1
        HiPress = drivepress > Pth2
        SlackGro = (th_dot - rReel*2*Ldot) < epsilon
        noSlack =  Lc < epsilon
\end{verbatim}

\texttt{LoPress} and \texttt{HiPress} determine when the eversion process stops and starts
respectively. \texttt{SlackGro} and \texttt{noSlack} define conditions for when
material crumples inside the housing between the reel and the eversion tube.

We then construct a table determining all transitions between the four possible states
(Table \ref{4x4stateTable}).


\begin{table}[h]
\begin{tabular}{c|c|c|c|c}
 To: \\From: &  Growing-Taut   & Growing-Slack   & Stuck-Taut  & Stuck-Slack \\ \hline
             &     0  & 1 & 2 & 3 \\ \hline
0    &  & SlackGro & LoPress  & LoPress {\bf and} SlackGro \\ \hline
1    & $L_c < \epsilon$  &   & LoPress \textbf{and} noSlack  &  LoPress \\ \hline
2    & HiPress    & HiPress \textbf{and} SlackGro &   &  SlackGro \\ \hline
3    & HiPress \textbf{and} $L_c<\epsilon$ & HiPress &   & \\
\end{tabular}\caption{4x4 state transition diagram deterimining growing or stuck and
taut or slack substates.  State in each row makes transitions to
column state when conditions
in each cell are met. Blank cells indicating in the same state.}\label{4x4stateTable}
\end{table}

%  Use name of bibliography files without .bib extension
%\bibliography{brl}

\section{Tubing Resistance}


\begin{figure}
\includegraphics[width=0.5\textwidth]{TubingResChart.png}
\caption{source: The Lee Company  \\ \url{https://www.theleeco.com/support-resources/engineering-tools/reference-information/tubing-flow/}}
\end{figure}



\end{document}

